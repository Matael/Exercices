\bexo

On considère le circuit électrique ci-dessous.

\begin{center}
\begin{tikzpicture}[scale=1, >=stealth]
	\draw (-.5,0) to[short, *-*] (5.5,0);
	\draw (-.5,2) to[short, i=$i$, *-] (0,2) to[R=$R$] (2,2) to[L=$L$] (4,2) to[short, -*] (5.5,2);
	\draw (4,2) to[C=$C$, *-*] (4,0);

	\draw[->] (-.5,.2) -- ++(0,1.6) node[midway, left] {$U_E$};
	\draw[->] (5.5,.2) -- ++(0,1.6) node[midway, right] {$U_S$};
\end{tikzpicture}
\end{center}

\begin{enumerate}
	\item Établir l'équation différentielle associée à l'évolution de $U_S$ dans le circuit
		en veillant à bien prendre en compte la tension d'entrée $U_E$.
	\item Résoudre l'équation différentielle et en dégager la solution générale dans le cas
		ou $\left(\nicefrac{R}{L}\right)^2 < \nicefrac{4}{LC}$. La solution particulière est à
		chercher sous la forme d'une constante.
	\item Rappeler quel système mécanique se comporte de manière similaire et indiquer la
		correspondance entre les composants électriques et les éléments mécaniques.
	\item En sachant que le système mécanique est régi par une équation de la forme :
		$x''(t) + 2\mu x'(t) + \omega_0^2x(t) = F$ indiquer quelle analogie est sous-entendue.
	\item Quel est l'avantage de disposer d'analogies électro-mécaniques ?
\end{enumerate}
\eexo
