\bexo

On considère un mouvement de chute libre dans un fluide visqueux. La position de la masse $m$ est notée $z(t)$. Cette fonction vérifie:
\begin{equation}\label{eq:ode_2}
mz''(t)+m\mu z'(t)=-mg.
\end{equation}

$\mu$ est une valeur de coefficient de frottement que nous ne connaissons pas pour le moment et que nous voulons déterminer. La masse a pour position initiale $z(0)=0$\,m et pour vitesse initiale $z'(0)=0\,$ m/s.\\

Nous allons trouver la valeur de la fonction $z(t)$ par deux méthodes différentes. La première consiste à trouver en un premier temps la valeur de la vitesse $v(t)$ de l'objet en chute libre lors de ce troisième cas puis à intégrer $v(t)$ pour trouver $z(t)$. La seconde méthode correspond à l'intégration directe de l'équation \refeq{eq:ode_2}, c'est-à-dire sans utiliser l'intermédiaire $v(t)$

\begin{itemize}
\item Donnez l'équation différentielle vérifiée par la fonction $v(t)$.
\item Donnez l'expression de $v(t)$
\item Intégrez cette expression et donnez une première expression pour $z(t)$
\item Donnez une solution particulière associée à \refeq{eq:ode_2}.
\item Quelle est la solution générale associée à cette équation. 
\item A l'aide de ces deux résultats, résolvez directement l'équation \refeq{eq:ode_2} et vérifiez que l'expression de $z(t)$ est bien la même que précédemment \end{itemize}

On observe qu'à l'instant $t=1$, on a $z(1)=-4$.

\begin{itemize}
\item Donnez une relation vérifiée par $\mu$.
\item Montrez que la valeur $\mu=0.71$ convient pour une valeur de $g=10$. (Hors barème)
\end{itemize}



\eexo
