\bexo

On considère deux mouvements de chute libre dans le champs de gravité $g$ en l'absence de frottements. L'équation du mouvement pour ces deux cas est la même et s'écrit:
\begin{equation}
mz''(t)=-mg.
\end{equation}

La valeur de l'accélération de pesanteur est égale à $10\,m/s^2$. Notons que l'axe du repère est orienté vers le haut. La première chute est repérée par une fonction $z_1(t)$. 

Elle a pour position initiale $z_1(0)=0\,m$ et pour vitesse initiale $z'_1(0)=1\,m/s$.  La seconde est repérée à l'aide de la fonction $z_2(t)$ a pour position initiale $z_2(0)=1\,m$ et une vitesse initiale nulle: $z'_2(0)=0\,m/s$.

\begin{itemize}
\item Donner les expressions des vitesses $z'_1(t)$ et $z'_2(t)$ et tracez les sur la même figure. 
\item Quelles similitudes peut on observer entre ces deux fonctions ? 
\item Existe t'il un ou plusieurs instants $t$ où ces deux fonctions sont égales ?
\item Donner les expressions de $z_1(t)$ et $z_2(t)$ et tracez les sur la même figure.
\item Quelles similitudes peut on observer entre ces deux fonctions ? 
\item Existe t'il un ou plusieurs instants $t$ où ces deux fonctions sont égales ?
\end{itemize}


\eexo
