\bexo

Beaucoup de matrices obtenues à l'aide de techniques de discrétisation sont au format creux (sparse en anglais). Cela signifie qu'elles sont pleines de zéros. Un format spécial a été crée. Il comporte trois vecteurs . \begin{itemize}
	\item un vecteur \tb{I} d'entiers correspondant à la ligne du coefficient \#$i$,
	\item un vecteur \tb{J} d'entiers correspondant à la colonne du coefficient  \#$i$, 
	\item un vecteur \tb{C} de nombre réels (ou complexes) correspondant à la valeur du coefficient \#$i$.
\end{itemize}
Ces trois vecteurs ont la même longueur qui correspond au nombre de coefficients non nuls de la matrice. Par exemple, la matrice suivante:
\begin{equation}
	[\tb{M}]=\begin{bmatrix}
		3.2 & 0 &0\\
		2.5&0 &1.23
	\end{bmatrix}
\end{equation}
est stockée de la façon suivante:
\begin{align}
	\tb{I}&=\begin{Bmatrix}
		1 & 2 & 2
	\end{Bmatrix}\\
	\tb{J}&=\begin{Bmatrix}
		1 & 1 & 3
	\end{Bmatrix}\\
	\tb{C}&=\begin{Bmatrix}
		3.2 & 2.5 & 1.23
	\end{Bmatrix}\\
\end{align} 

\begin{itemize}
	\item Quels sont les trois vecteurs associés à la matrice:
\begin{equation}
	[\tb{M}_1]=\begin{bmatrix}
		2 & -1 \\
		-1&2 &-1\\
		&\ddots&\ddots &\ddots \\
		&&-1&2&-1\\
		 &&&-1&2
	\end{bmatrix}?
\end{equation}
Quel est l'avantage de ce système de stockage pour des matrices de ce type ? Illustrer votre réponse.
\item Ecrire une fonction Matlab/Octave
\begin{verbatim}
	dimension_sparse(I,J,C)
\end{verbatim}
où $I$, $J$ et $C$ sont les trois vecteurs définissant une matrice $[\tb{M}]$ au format sparse. Cette fonction devra retourner la dimension de la matrice (nombre de lignes et de colonnes). 
\item Ecrire une fonction Matlab/Octave
\begin{verbatim}
	muliply_sparse(I,J,C,X)
\end{verbatim}
où $I$, $J$ et $C$ sont les trois vecteurs définissant une matrice $[\tb{M}]$ au format sparse. X est un vecteur de taille adéquate. Cette fonction devra retourner le produit de $[\tb{M}]$ par ${X}$.
\item Ecrire une fonction qui multiplie deux matrices au format sparse entre elles.
\end{itemize}




\eexo
\solution{
}
