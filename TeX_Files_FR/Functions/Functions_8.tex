\bexo
Soit une fonction $f:x\mapsto f(x)$. Expliquer comment déduire du graphe de la fonction $f$ celui de
\begin{itemize}
	\item $x\mapsto 2f(x)$
	\item $x\mapsto f(x+4)$
	\item $x\mapsto f\left(2(x-1)\right)$
\end{itemize}

	\ifsolutions \else
\vspace*{8cm}
\fi
\eexo
\solution{
\vspace*{0.3cm}
\begin{itemize}
	\item $x\mapsto 2f(x)$: Cette fonction est trois fois plus importante que la fonction $f$. On effectue donc une affinité d'axe $0x$ et de rapport 2.
	\item $x\mapsto f(x+4)$: Cette fonction est la fonction $f$ décalée de $4$ vers la gauche.
	\item $x\mapsto f\left(2(x-1)\right)$. On doit effectuer deux opérations successives.  D'abord décaler de $1$ vers la droite le graphe de la fonction $f$ puis contracter l'axe des $x$ par deux (affinité d'axe $Oy$ et de rapport 2, la fonction est donc "deux fois moins étalée").
\end{itemize}


}
