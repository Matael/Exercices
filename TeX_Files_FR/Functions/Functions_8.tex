\bexo
Soit une fonction $f:x\mapsto f(x)$. Expliquer comment déduire du graphe de la fonction $f$ celui de 
\begin{itemize}
	\item $x\mapsto f(3x)$
	\item $x\mapsto 3f(x)$
	\item $x\mapsto f\left(\dfrac{x+1}{2}\right)$
\end{itemize}
	
	\ifsolutions \else 
\vspace*{8cm}
\fi 
\eexo
\solution{
\vspace*{0.3cm}
\begin{itemize}
	\item $x\mapsto f(3x)$: Cette fonction avance "3 fois plus vite" que la fonction $f$. Cela revient donc à contracter l'axe des abscisses. Ainsi la fonction $f(3x)$ vaudra en $1$ ce que la fonction $f$ vaudra en trois. On effectue une affinité d'axe $0y$ de rapport $1/3$.  
	\item $x\mapsto 3f(x)$: Cette fonctions est trois fois plus importante que la fonction $f$. On effectue donc une affinité d'axe $0x$ et de rapport 3. 
	\item $x\mapsto f\left(\dfrac{x+1}{2}\right)$. On doit effectuer deux opérations successives. D'abord décaler de $1$ vers la gauche le graphe de la fonction $f$ puis dilater l'axe des $x$ par deux (affinité d'axe $Oy$ et de rapport 2)
\end{itemize}


}