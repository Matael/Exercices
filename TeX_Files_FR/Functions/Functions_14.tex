\bexo
\begin{itemize}
	\item On considère une course de relais de 300 m avec trois coureurs parcourant chacun 100 m. Chaque coureur court à vitesse constante:
	\begin{itemize}
		\item Le premier à $100\,$ m/s
		\item Le second à $100\,$ m/s
		\item Le troisième à $1\, $m/s
\end{itemize}
	Calculer la vitesse moyenne de la course par deux moyens différents.		 
	\item On considère cette fois-ci une course de 100 m avec un seul coureur. Celui-ci court à accélération constante et parcourt les 100 m en 1 s. Quelle est l'expression de sa vitesse en fonction du temps ?  
\end{itemize}

\eexo


