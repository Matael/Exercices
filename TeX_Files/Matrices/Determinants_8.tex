\bexo
Given

\begin{equation}
	\mat{A} = \begin{bmatrix}
		1& 0& 0& 0\\
		0& 1& 0& 1\\
		1& 0& 1& 1\\
		2& 3& 1& 1
	\end{bmatrix}
	~~,~~
	\mat{P}_1 = \begin{bmatrix}
		0& 0& 1& 0\\
		0& 0& 0& 1\\
		1& 0& 0& 0\\
		0& 1& 0& 0
	\end{bmatrix}
	~~,~~
	\mat{P}_2 = \begin{bmatrix}
		1& 0& 0& 0\\
		0& 0& 0& 1\\
		0& 0& 1& 0\\
		0& 1& 0& 0
	\end{bmatrix}
\end{equation}

evaluate:

\begin{enumerate}
	\item $\mathrm{det}(\mat{A})$
	\item $\mathrm{det}(\mat{P}_1\mat{A})$
	\item $\mathrm{det}(\mat{P}_2\mat{A})$
\end{enumerate}

Can you deduce a more general rule for the determinants of permutated matrices?
\eexo

\solution{
\hfill\\
\begin{enumerate}
	\item $-3$
	\item $-3$
	\item $3$
\end{enumerate}

The permutation matrix has a determinant of $(-1)^N$ with $N$ the number of permutations.
}

