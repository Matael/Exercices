\bexo
Given
\begin{equation}
	A = \begin{bmatrix}
		1& 0\\ 1& 2
	\end{bmatrix}
	~~,~~
	B = \begin{bmatrix}
		7& 9\\ 0& 2
	\end{bmatrix}
	~~,~~
	C = \begin{bmatrix}
		5& 6\\ 4& 3
	\end{bmatrix}
	~~,~~
	D = \begin{bmatrix}
		2& 4\\ 5& 6
	\end{bmatrix}
\end{equation}

and $\mathbf{0}_n$ a $n\times n$ matrix of zeros;

evaluate

\begin{equation}
\det{\begin{matrix}
	\mat{A}& \mathbf{0}_2\\
	\mathbf{0}_2 & \mat{D}\\
\end{matrix}}
~~,~~
\det{\begin{matrix}
	\mat{A}& \mat{B}\\
	\mathbf{0}_2 & \mat{D}\\
\end{matrix}}
~~,~~
\det{\begin{matrix}
	\mat{A}& \mathbf{0}_2\\
	\mat{C} & \mat{D}\\
\end{matrix}}
\end{equation}

What rule(s) can you deduce from the previous calculations?
\eexo


\solution{
	The three matrices have the same determinant ($-16$), the extra-diagonal \textit{block} does not come in play.
	The following equivalence then holds:
	\begin{equation}
		\det{\begin{matrix}
			\mat{A}& \mat{B}\\
			\mathbf{0}_2 & \mat{D}\\
		\end{matrix}}
		=
		\det{\begin{matrix}
			\mat{A}& \mathbf{0}_2\\
			\mat{C} & \mat{D}\\
		\end{matrix}}
		=
		\mathrm{det}(\mat{A})
		\mathrm{det}(\mat{D})
	\end{equation}
}
