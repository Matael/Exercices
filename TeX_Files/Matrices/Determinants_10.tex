\bexo
Given

\begin{equation}
	\mat{M} = \begin{bmatrix}
		1& 0& 0& 0\\
		0& 1& 0& 1\\
		1& 0& 1& 1\\
		2& 3& 1& 1
	\end{bmatrix}
	~~,~~
	\mat{N} = \begin{bmatrix}
		1& 1& 0& 3\\
		3& 4& 1& 3\\
		1& 1& 0& 1\\
		1& 0& 0& 3
	\end{bmatrix}
\end{equation}

Evaluate:

\begin{enumerate}
	\item $\mathrm{det}(\mat{M})$
	\item $\mathrm{det}(\mat{N})$
	\item $\mathrm{det}(\mat{M}\mat{N})$
	\item $\mathrm{det}(\mat{N}\mat{M})$
\end{enumerate}

What rule(s) can you derive from the previous calculations?
\eexo

\solution{
\hfill\\
	\begin{enumerate}
		\item $\mathrm{det}(\mat{M}) = -3$
		\item $\mathrm{det}(\mat{N}) = 2 $
		\item $\mathrm{det}(\mat{M}\mat{N}) = -6$
		\item $\mathrm{det}(\mat{N}\mat{M}) = -6$
	\end{enumerate}

	The determinant of a product is the product of the determinants.
}
