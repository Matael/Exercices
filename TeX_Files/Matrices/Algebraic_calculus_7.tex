\bexo
The Hooke's matrix $\mat{C}$ for 
a material relates the vector of strain $\bs{\eps}$ to the vector of stresses $\bs{\sigma}$. 
\begin{equation}
	\bs{\sigma}=\mat{C}\bs{\eps}.
\end{equation}
For an isotropic material, the Hooke's matrix depends on only two coefficients called the Lam\'e coefficients $\lambda$ and $\mu$ and
\begin{equation}
	\mat{C}=
	\begin{bmatrix}
		\lambda+2\mu & \lambda & \lambda &0&0&0\\
		 \lambda &\lambda+2\mu & \lambda &0&0&0\\
		 \lambda &\lambda &\lambda+2\mu & 0&0&0\\
		 0&0&0&2\mu &0&0\\
		 0&0&0&0&2\mu &0\\
		 0&0&0&0&0&2\mu  
	\end{bmatrix}
\end{equation}
The Lam\'e coefficients $\lambda$ and $\mu$ can be expressed from the young's modulus $E$ and Poisson coeficient $\nu$ by 
\begin{equation}
	\lambda=\dfrac{1+\nu}{1-2\nu}\esp \mu=\dfrac{E}{2(1+\nu)}
\end{equation}


The compliance matrix $\mat{S}$ is defined as the inverse of $\mat{C}$. \\

What is the value of the compliance matrix as a function of $E$ and $\nu$?


\eexo 

\solution{
\begin{equation}
	\mat{S}=\dfrac{1}{E}\begin{bmatrix}
		1 & -\nu & -\nu &0 &0&0\\
		-\nu & 1 & -\nu &0 &0&0\\
		-\nu & -\nu &1 & 0 &0&0\\
		0 &0 &0&1+\nu &0 &0\\ 
	0&	0 &0 &0&1+\nu &0\\ 
0&0&0 &0 &0&1+\nu \\ 
	\end{bmatrix}
\end{equation}



}
	




