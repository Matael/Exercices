\bexo
What are the roots of  $X^3+26X^2+23X-50 $ ?
\eexo
\solution{
1 is one root of the polynomial. It is then possible to factorize by $(X-1)$. This can be done by two means. \\

The first one is euclidian division
\begin{equation}
	\begin{array}{r|l}
	{X^3+26X^2+23X-50}&{X-1}\\	 \hline
	{27X^2+23X-50} &{X^2+27X+50}\\
	{50X-50}&
	\end{array}
\end{equation}


The second method is the identification 

\begin{align}
	X^3+26X^2+23X-50&=(X-1)(aX^2+bX+c)\\
	&=aX^3+(b-a)X^2+(c-b)X-c
\end{align}
By identification of the coefficients in $X^3$:
\begin{equation}
	a=1.
\end{equation}
By identification of the coefficients in $X^2$:
\begin{equation}
	b-a=26 \so b=27.
\end{equation}
By identification of the coefficients in $X$:
\begin{equation}
	c-b=23 \so c=50.
\end{equation}
We should check that this value of $c$ is Ok and leads to an identification of the term of zero order. This is the case, and an easy way to check if the calculus is well done. 

The two other roots of out polynomial are the one of $X^2+27X+50$. Once more, two methods to find the result. \\

The first one is the classical one
\begin{equation}
	\Delta=27^2-4\times 50=529=23^2
\end{equation}

The two roots are then
\begin{equation}
	\dfrac{-27±\pm\sqrt{\Delta}}{2}\so X=-25 \tr{ and } X=-2.
\end{equation}

The second method corresponds to the formula
\begin{equation}
	(X-a)(X-b)=X^2-(a+b)X+ab
\end{equation}
Hence $a+b=-27$ and $ab=50$. The two roots are then $-25$ and $-2$.

}